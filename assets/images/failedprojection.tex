\documentclass[tikz]{standalone}
\usepackage{tikz}
\usetikzlibrary{intersections,calc}
\begin{document}
\begin{tikzpicture}[scale=0.2]

% Helpers
\def\drawmaybe{draw}
% Size definitions
\def\dotsize{6pt}
\def\eyediameter{2.4}
% model eye ball center
\def\ex{-3.95}
\def\ey{3.27}
% image pupil
\def\projdistance{45}
\def\px{\projdistance}
\def\py{2.6}
% screen
\def\sx{\projdistance}
\def\sy{-5}
\def\sh{20.7}
\def\sangle{20}
% imagined projection
\def\prox{\ex+0.5*\eyediameter}
\def\proy{-10}

% Clip to only important area
\clip (-10, -25) rectangle (50, 8);
% Draw pupil
\coordinate[label=right:{$p$}] (pupil) at (\px, \py);
\fill[black] (pupil) circle (\dotsize);

% Draw eye ball
\coordinate[label=left:{$e$}] (eyeballcenter) at (\ex, \ey);
\fill[black] (eyeballcenter) circle (\dotsize);
\draw[name path=eyeball] (eyeballcenter) circle (\eyediameter);

% Define screen and projection plane
\draw[draw, name path=screen] (\sx, \sy) -- node[sloped,below] {screen} +(270 - \sangle: \sh);
\path[name path=projection, gray] (\prox, \proy) -- node[right, near end] {$t_a$} +(0, 30);

% Draw projection of pupil
\draw[draw, name path=pupilprojection, red!50] (pupil) -- +(-1.5 * \projdistance, 0);

% Calculate intersection points
% Storing coordinates as suggested in https://tex.stackexchange.com/a/122245/51583
\path[name intersections={of=pupilprojection and projection, by={x}}] coordinate (p0) at (x);
\path[name intersections={of=pupilprojection and eyeball, by={,y}}] coordinate (p1) at (y);

% Calculate rays from eyeball through intersections
% This seems to work but without proper coordinates to draw points:
\path[name path=rayx,draw,gray,style={shorten >= -15cm}] (eyeballcenter) -- (p0);
\path[name path=rayx,draw,gray,style={shorten >= -15cm}] (eyeballcenter) -- (p1);
\path[name path=ray0] (eyeballcenter) -- (p0);
\path[name path=ray1] (eyeballcenter) -- (p1);

% Intersect rays with screen
% TODO: coordinates x are just for debugging, although it is surprising that there two intersections with the circle
\path[name intersections={of=ray0 and screen, by={x,y}}] coordinate (s0b) at (x) coordinate (s0) at (y);
\path[name intersections={of=ray1 and screen, by={x,y}}] coordinate (s1b) at (x) coordinate (s1) at (y);

% Draw intersections (last to be on top of the lines)
\fill[red] (p0) circle (\dotsize) node[label={[label distance=0pt]275:$p_0$}] {};
\fill[blue] (p1) circle (\dotsize) node[label={[label distance=0pt]30:$p_1$}] {};

% Draw intersections on screen
%\fill[purple] (s0) circle (\dotsize) node[above] {$s_0$};
%\fill[blue] (s1) circle (\dotsize) node[below] {$s_1$};

% TODO: From here only debugging. In https://tex.stackexchange.com/a/147144/51583 it's similar and seems to work... hmhm
%\fill[purple] (s0b) circle (\dotsize) node[above] {$s_0b$};
%\fill[blue] (s1b) circle (\dotsize) node[below] {$s_1b$};
%\path[name intersections={of=pupilprojection and screen, by={w,}}] coordinate (la) at (w);
%\fill[green] (w) circle (\dotsize) node[below] {$w$};

\end{tikzpicture}
\end{document}
